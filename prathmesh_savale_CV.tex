%
% LaTeX source of my resume
% =========================
%
% Heavily commented to to fit even LaTeX beginners (hopefully).
%
% See the `README.md` file for more info.
%
% This file is licensed under the CC-NC-ND Creative Commons license.
%
% Modified heavily by Prathmesh Savale for personal use
% All credit goes to the original author

% Start a document with the here given default font size and paper size.
\documentclass[10pt,a4paper]{article}

% Set the page margins.
\usepackage[a4paper,margin=0.75in]{geometry}

% Setup the language.
\usepackage[english]{babel}
\hyphenation{Some-long-word}

% Makes resume-specific commands available.
\usepackage{resume}

\usepackage{enumitem}


\begin{document}  % begin the content of the document
\sloppy  % this to relax whitespacing in favour of straight margins


% title on top of the document
\maintitle{Prathmesh Savale}{}{}

\nobreakvspace{0.3em}  % add some page break averse vertical spacing

% \noindent prevents paragraph's first lines from indenting
% \mbox is used to obfuscate the email addressL
% \sbull is a spaced bullet
% \href well..
% \\ breaks the line into a new paragraph
\noindent
\small{
\href{https://praths007.github.io/}{https://praths007.github.io/}\sbull
\href{https://github.com/praths007}{https://github.com/praths007}\sbull
\href{mailto:prathmesh.savale@gmail.com}{prathmesh.savale@gmail.com}\sbull
+91 8087744932 }


\spacedhrule{0.9em}{-0.4em}  % a horizontal line with some vertical spacing before and after

%--------------------------------------------------------------------------------------------------------------------------
%\roottitle{Summary}  % a root section title
%
%\vspace{-1.3em}  % some vertical spacing
%\begin{multicols}{2}  % open a multicolumn environment
%\noindent \emph{Creative geek with roots in the open source movement, an entrepreneurial mindset and a passion for delivering value by developing maintainable software.}
%\\
%\\
%At the age of seven (1989) Cies wrote his first lines of code in a \acr{LOGO}-like language on an \acr{MSX} (pre-\acr{PC}).  Two years later he attended a conference on an emerging new technology, the Internet, at the Erasmus University from which he would graduate 16 years later.
%
%After being introduced to the open source movement in 1997, he taught himself a variety of skills including system administration and programming (Bash, Python, Ruby \& \CPP).  By 2002 he got his pet project \acr{KT}urtle ---a zero-entry-barrier programming environment--- included into \acr{KDE}'s \emph{edu} module, and thereby almost every Linux distribution.
%
%From 2003 to 2007 he studied at the Erasmus University Rotterdam and graduated in \emph{Business and Computer Science} (one curriculum).  After graduation he travelled Europe and Asia during a two year sabbatical, on which he ``hustled'' several IT gigs (see experiences below) to extend the journey.
%\end{multicols}
%
%
%\spacedhrule{0em}{-0.4em}

%--------------------------------------------------------------------------------------------------------------------------
% Experience
%--------------------------------------------------------------------------------------------------------------------------
\roottitle{Experience}

\headedsection  % sets the header for the section and includes any subsections
  {{Kiewit Corporation}}
  {\textsc{Bangalore, India}} {%
  \headedsubsection
    {Data Analyst}
    {Oct \apo18 -- present} {%
	\inhouseprojectsection 
	 {Predictive maintenance of Caterpillar haul trucks at coal mines} {%
		{\bodytext{
		\begin{itemize}
		\item Built a classification framework using \acr{LSTM} networks to predict breakdown of 
		Caterpillar haul trucks at Kiewit's coal mining facility. Accurate prediction helps save additional cost of repairing an
 		unplanned breakdown and increases throughput of the mining facility.
		\item Improved the \acr{AUROC} by 20\% using rolling, tumbling and hopping aggregates on the truck's sensor 			data. Used \acr{SHAP} values to identify  features that contribute to the maximum number of breakdowns.
		\end{itemize}
		}}
	}
	\inhouseprojectsection 
	{Forecasting gasoline and electricity consumption for private vehicles} {%
		{\bodytext{
		\begin{itemize}
		\item Estimated depreciation of gasoline consumption and subsequent increase
		of electricity consumption due to the introduction of electric vehicles
		across all states in the \acr{US}, using \acr{ARIMA} and random walk models.
		\item Created a parser in \acr{R} to extract relevant information such as vehicle miles traveled, fuel consumption, vehicle type, 			etc. from \acr{CSV} flat-files that are procured from various government and public survey websites.
		\end{itemize}
		}}
	}
%	\inhouseprojectsection 
%	{Optimizing the fleet size of vehicles at construction sites} {%
%		{\bodytext{
%		\begin{itemize}
%		\item Created a framework using regression and linear reward inaction to
%		determine the optimum fleet size of vehicles at construction sites which helps reduce vehicle idle time and maintenance 		costs.
%		\end{itemize}
%		}}
%	}
}
}

\headedsection  % sets the header for the section and includes any subsections
  {{Mu Sigma Inc.}}
  {\textsc{Bangalore, India}} {%
  \headedsubsection
    {Decision Scientist}
    {Sep \apo15 -- Oct \apo18} {%
	\inhouseprojectsection 
	 {Building sales forecasting framework | Client - \acr{UK}'s largest retailer} {%
		{\bodytext{
		\begin{itemize}
		\item Built a forecasting framework using \acr{ARIMA} with seasonal adjustment to produce forecasts at different levels
		of the buying hierarchy. It is used by commercial teams for budgeting and inventory management.
		\item Included adjustment for external regressors like holidays along with Yeo-Johnson transformation for normalization which helped improve the overall company level forecast accuracy by 5.6\%. 
		\item Parallelized parameter tuning, model building, scoring, and forecasting for \char`\~2500 stores and
		\char`\~3600 products using PySpark. Used test-driven development to ensure error free codebase in a \acr{CI}/\acr{CD} pipeline.
		\end{itemize}
		}}
	}
	\inhouseprojectsection 
	{Reducing device failure rates | Client - Fortune 3 technology company} {%
		{\bodytext{
		\begin{itemize}
		\item Created a boosted trees ensemble to predict electronic device failures
		leading to a 3\% reduction (9\% to 6\%) in failure rate which translates to a cost reduction of \char`\~1.8 million \acr{USD}
		annually in inventory management.
		\item Implemented cascading classifiers to decrease collateral damage while
		predicting device failures.
		\item Completely automated and deployed the analytical solution using Jenkins saving \char`\~40 man-hours each week.
		\end{itemize}
		}}
	}

}
}

\headedsection  % sets the header for the section and includes any subsections
  {{Persistent Systems}}
  {\textsc{Pune, India}} {%
  \headedsubsection
    {Engineering Intern}
    {Jun \apo14 -- May \apo15} {%
	\inhouseprojectsection 
	 {Developing \acr{CUDA} based image processing application | Internship} {%
		{\bodytext{
		\begin{itemize}
		\item Developed a \acr{CUDA C} based application to execute a computationally expensive content-aware image resizing algorithm called seam carving on \acr{GPU}.
		 This helped achieve 7.5X acceleration in execution time over traditional \acr{CPU} execution due to the high degree of parallelism of Nvidia propriety \acr{CUDA} based \acr{GPU}s.
		\end{itemize}
		}}
	}
}
}

\spacedhrule{0.9em}{-0.4em}


%--------------------------------------------------------------------------------------------------------------------------
% Education
%--------------------------------------------------------------------------------------------------------------------------

\roottitle{Education}

\headedsection
  {\href{}{University of Pune}}
  {\textsc{Pune, India}} {%
  \headedsubsection
    {Bachelor degree in Computer Engineering}
    {2011 -- 2015}
    {\bodytext{Completed with a First Class with Distinction grade. Ranked 3rd in a class of 180 students based on cumulative scores.}}
}

\spacedhrule{0.9em}{-0.4em}

%--------------------------------------------------------------------------------------------------------------------------
% Skills
%--------------------------------------------------------------------------------------------------------------------------

\roottitle{Skills}

\inlineheadsection  % special section that has an inline header with a 'hanging' paragraph
 {Programming:}
{Python and PySpark, C and C++, R, SQL and HiveQL, Bash}
%  {Programming:}
%  {Software design and implementation, with(in) a team.  Big fan of Agile methodologies (Scrum and Kanban), automated deployment (Capistrano) and continuous integration (Hudson/Jenkins).  Enjoys writing Ruby/\nsp Python/\nsp Java/\nsp \CPP, yet flirts regularly with Haskell.  Solid knowledge of web technologies:\ \acr{HTML+CSS}, \acr{XML}, \acr{RDF}, \acr{REST}, \acr{SOAP} and JavaScript (mostly Angular and jQuery).  Linux administration skills:\ Bash, Apache, My\acr{SQL}, Postgres\acr{SQL}, virtualization/cloud (Vagrant, Open\acr{VZ}, \acr{VM}ware, \acr{KVM}, Xen and \acr{EC}2), datacenter automation (Puppet and Chef).}
\vspace{0.5em}

\inlineheadsection
 {Computational programming:}
{octave, numpy, pandas, tidyverse, seaborn, ggplot2, scikit-learn, statsmodel, keras}
\vspace{0.5em}

\inlineheadsection
 {Statistical Analysis:}
{Regression, Bagging, Boosting, Ensemble, Hypothesis testing}
\vspace{0.5em}

\inlineheadsection
  {Tools:}
  {iPython and Google colab, Teradata, Spark and Hadoop, Git and Github, \LaTeX, Jenkins, Jira}

\spacedhrule{1.6em}{-0.4em}

%--------------------------------------------------------------------------------------------------------------------------
% Certifications
%--------------------------------------------------------------------------------------------------------------------------

\roottitle{Certifications}

\inlineheadsection
  {Machine Learning:}
  {Audited Coursera MOOC by Andrew Ng. \href{https://www.coursera.org/account/accomplishments/certificate/6UDGCV3A6A7L?utm_medium=certificate&utm_source=link&utm_campaign=copybutton_certificate}{[Certificate]}
\href{https://github.com/praths007/coursera_machine_learning}{[Code]}}

\vspace{0.5em}

\inlineheadsection
  {Decision Scientist:}
  {Audited credit based certification course by Mu Sigma. \href{https://www.portal.mu-sigma.com/msu/DecisionScientistCertificate/8388-Jun2017-10390.html}{[Certificate]}}

\vspace{0.5em}

\inlineheadsection
  {Machine Learning A-Z: Hands-On Python \& R In Data Science:}
  {Udemy MOOC. \href{https://www.udemy.com/certificate/UC-PXVDG31R/}{[Certificate]}}

\spacedhrule{1.6em}{-0.4em}

%--------------------------------------------------------------------------------------------------------------------------
% Personal Projects
%--------------------------------------------------------------------------------------------------------------------------

\roottitle{Personal Projects}

\inlineheadsection
  {Tech support call logger:}
  {Minimalist application which uses google speech \acr{API} to transcribe call logs and \acr{LIUM} speaker diarization to assign speaker identity to the transcribed text at tech support call centers. \href{https://github.com/praths007/speechtotext/}{[Code]}}

\vspace{0.5em}

\inlineheadsection
  {Shopping cart tracker:}
  {Application to track route of shopping carts in malls using kalman filters. An ibeacon sensor is attached to every shopping cart. Three receiver sensors strategically placed in the mall, correctly triangulate the position of the ibeacon enabled cart in realtime. \href{https://github.com/praths007/iot_tracking_shopping_carts/}{[Code]}}


\end{document}

＀
